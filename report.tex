\documentclass[a4paper, titlepage, 12pt, draft]{article}
\usepackage[utf8]{inputenc}
\usepackage{times}
\usepackage{longtable}

\linespread{1.4} % this amounts to 1.5 linespacing, for odd reasons.

%redefine percentage sign to be a little smaller
\let\oldpct\%
\renewcommand{\%}{\scalebox{.9}{\oldpct}}

\begin{document}

\title{Uncalibrated eye-typing facilitated by probabilistic language and input modelling
\\---\\
A report for the Thesis Preparation course, fall semester 2016}
\author{
	Sigurt Bladt Dinesen
	\\\texttt{sidi@itu.dk}
}

\maketitle

\section*{Caveat}
I have applied for an exemption to postpone my thesis to the fall semester of
2017. As of February 2017, my application was granted by the Student Affairs \&
Programmes I can't quite avoid that this report be affected by this delay. For
example, I have not registered for the project yet. If credits are to be
registered for supervising me up to this time, it should go to Dan Witzner
Hansen, who is also supervising me in another project, that this project will build on.

\section*{Introduction}
The purpose of this project is to build an eye typing system, with uncalibrated eye tracking.

Typing is a ubiquitous input method for human-computer interaction, particularly for
human-to-human communication facilitated by computers.

Several methods have been developed to help humans type faster and more
accurately, especially on mobile platforms like mobile phones. Some methods are
based on probabilistic language- and input-modelling, such as T9 and the
replace-as-you-type technologies (colloquially; \textit{autocorrect}) that have mostly
replaced T9 on contemporary smartphones, perhaps due to widespread deployment of
touchscreens. Other methods use gestures, or leverage creative arrangements of
input symbols. Examples include swype for Android and IOS, and Dasher and
StarGazer intended for use with eye tracking.

The use of eye tracking as a means of text input (eye typing) can be demanding
on the user. Exact and deliberate control of the gaze is difficult and tiring,
and in addition to the energy expended when typing, most systems require
users to go through calibration routines before use.

One area where eye typing has seen use is as a communication platform for ALS
(amyotrophic lateral sclerosis) patients. Patients suffering suffering from ALS
often retain the use of their eye muscles longer than that of other muscles, and
eye typing can therefore help increase their quality of life. In practice,
patients patients tend to use eye typing for short commands and answers. For
such usage patterns, uncalibrated eye typing may be particularly beneficial.

This project builds on a ongoing project that I am doing at the ITU, and so the
details of this project cannot be fully determined yet. To make things simpler,
this report will regard the ongoing project as "previous work" that has not yet
been studied, and can hence affect the outcome of, and methods used in, this
project, but not the goal and overall focus.

The following outlines the methods I plan to investigate, to achieve the project
goal:

\begin{enumerate}
\item A language- and input-model will be designed and implemented, in order to
make educated guesses as to the intended user input, in an attempt to compensate
for tracker (and user) imprecisions. The concrete design of he model
research-to-be-done, but is envisioned as a combination of the following:
\begin{itemize}
    \item Interpreting the geometrical user input (points on a plane) as words with a
    probabilistic word search.
    \item Limiting the need for typing, by using predictive word-completion, in a
      manner similar to that used in T9.
\end{itemize}
\item A typing system will be implemented, implementing uncalibrated tracking on
top of existing tracker hardware.
3. The typing system(s) will be
evaluated with a typing performance test in a small-scale user-study.
\end{enumerate}

The deliverables of is project are software implementations of the language- and
input-model, of the eye typing system(s), and a report detailing the project
process and results.

On the surface, this project is similar to my aforementioned previous project. But while the
previous project is a small (7.5 ects points), shallow exploration of numerous
solution spaces, this projects aims to go in depth in producing a single
solution to the problem of uncalibrated eye-typing. It is hence a separate
project, in the same setting.

\section*{Work so far}
My work so far has consisted mostly in talking to supervisors, and studying
literature. I have delayed my thesis because I was unsure what I wanted to do
for my thesis. I still don't, really. This makes it hard to write this report,
but I have chosen to focus on one of the projects I have in mind. This means
however, that a good deal of time has been spend merely considering different
projects, and talking to potential supervisors with, no product to show for it.
I realise that this makes for a weak report, but hope that my expended effort
is apparent enough to make the report acceptable. I have considered a total of 3
different paths:

\begin{enumerate}
	\item Uncalibrated eye typing (the one described in this report).
	\item Projects relating to locality, or even distance, sensitive
	hashing, with application to similarity search. (discussed with Francesco
	Silvestri, and breifly with Rasmus Pagh)
	\item Projects relating to GPU-computations of insurance-solvency, and
	a domain specific language for modelling insurance contracts. (Discussed with Peter Sestoft)
\end{enumerate}

These are all continuations of, or related to projects I have previously done
on my masters at ITU.

For the uncalibrated eye typing project, my reading includes: Some chapters from
yarbus\cite{yarbus}, as early seminal work on gaze tracking and attention
modelling. Augustin et al.\cite{witzner}, using a low-cost calibrated
gaze-tracker for eye typing. Esteves et al. \cite{orbits}, evaluating a
calibration-free eye typing system, using moving "orbits" and Pearson
correlation to compensate for the lack of calibration. Pfeuffer et
al.\cite{pursuit}, introducing the use of moving targets along with Pearson
correlation, but using it to make the calibration processes less tedious, not to
climate it as in \cite{orbits}. Finally, a few chapters (mostly chapter 6) of
Marajanta's dissertation from 2009\cite{majaranta} covers the subject of eye
typing in general. Chapter 6 and 7 specifically deal with the use of language
and input modelling.

\section*{Project plan}
\subsection*{Report Outline}
Following is a rough outline of the project report.
\begin{enumerate}
	\item Introduction
	\begin{enumerate}
		\item Problem Statement
		\item Background and Previous Work
	\end{enumerate}
	\item Method
	\begin{enumerate}
		\item Language Modelling
		\item Input Modelling
	\end{enumerate}
	\item User Study
	\begin{enumerate}
		\item Study Description
		\item Study Results
	\end{enumerate}
	\item Project Conclusion
	\begin{enumerate}
		\item Project Summary
		\item Implications / Possible Future Work
	\end{enumerate}
\end{enumerate}

\subsection*{Detailed plan}
Following is a (mostly) week-by-week plan to complete the project.
\begin{longtable}[hbt]{|p{7cm}|p{5cm}|}
	\hline
	 What (agreements, deadlines, working tasks) &
	 When (allocated time slots, deadlines) \\
	 \hline
	 \hline
	 Finish literature studies & Week 29-33. \\
	 \hline
	 Finalize project agreement & Week 34. \\
	 \hline
	 Semester officially starts & 28. Aug. \\
	 \hline
	 Submit final project agreement to board of studies & 29. Aug. \\
	 \hline
	 Supervisor meetings held every other week & Starting with week 36\\
	 \hline
	  Write introduction sections for the report.
	  Collate datasets for language models & week 37\\
	 \hline
	  Implement and finalize languages model & week 38\\
	 \hline
	  Setup physical/hardware aspects of eye tracking system, implement
	 software abstraction around it. Write about the background of language and
	 input modelling techniques in the method section for the report & week 39\\
	 \hline
	  Implement uncalibrated eye tracking, based on Pearson-correlation & week 40\\
	 \hline
	  Implement input model. Organize user-study (find subjects) & week 41\\
	 \hline
	  Buffer week for making sure implementations are robust enough for
	  user studies, fixing implementation errors, and detail. Also
	  implement log/data-gathering to facilitate user study & week 42\\
	 \hline
	  Setup and perform user study, collate data and make conclusions & week 43\\
	 \hline
	  Finish user study sections in report & week 44\\
	 \hline
	  Write detailed description of the implemented language and input
	  models for the method section in the report & week 45\\
	 \hline
	  Write project conclusion section of report & week 46\\
	 \hline
	  Polish and proof read full report& week 47\\
	 \hline
	  Buffer week for unforeseen problems, because I've always wished I'd had one in other projects & week 48\\
	 \hline
	 Project deadline & 2. Jan. \\
	\hline
\end{longtable}

\begin{thebibliography}{9}
\bibitem{yarbus}
Alfred L. Yarbus.Eye Movements and Vision.PLENUM PRESS - NEW YORK, 1967.

\bibitem{witzner}
Javier san Agustin, Emilie Mollenbach, Maria Barret, Martin Tall, Dan Witzner
Hansen, and Paulin John Hansen. Evaluation of a low-cost open-source gaze
tracker. 2010.


\bibitem{orbits}
Augusto Esteves, Eduardo Velloso, Andreas Bulling, and Hans Gellersen. 2015.
		Orbits: Gaze Interaction for Smart Watches using Smooth Pursuit
		Eye Movements. In Proceedings of the 28th Annual ACM Symposium
		on User Interface Software \& Technology (UIST '15). ACM, New
		York, NY, USA, 457-466. DOI:
		https://doi.org/10.1145/2807442.2807499

\bibitem{pursuit}
Ken Pfeuffer, Melodie Vidal, Jayson Turner, Andreas Bulling, and Hans
Gellersen. 2013. Pursuit calibration: making gaze calibration less tedious and
more flexible. In Proceedings of the 26th annual ACM symposium on User
interface software and technology (UIST '13). ACM, New York, NY, USA, 261-270.
DOI: http://dx.doi.org/10.1145/2501988.2501998
\bibitem{majaranta}
Päivi Majaranta. Text entry by eye gaze. 2009. (dissertation)

\end{thebibliography}
\end{document}
